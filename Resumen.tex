%!TeX root=../thesisStructure.tex
\chapter*{Resumen} % Abstract for master degree thesis
\addcontentsline{toc}{chapter}{Resumen}

A nivel mundial, existe permanentemente un problema sanitario relacionado con la carencia de agua de calidad para consumo humano. Este problema afecta principalmente y de manera más severa a aquellos países con un desarrollo 
económico y social bajo, sin embargo, incluso los países de más alto nivel de desarrollo llegan a presentar este tipo de problema cuando sus procesos de tratamiento de agua no cumplen con los estándares establecidos por las 
instituciones regulatorias.

La Calidad del Agua hace referencia a los rangos de valores requeridos para los parámetros o variables fisicoquímicos y bacteriológicos que influyen en los niveles de contaminación de una muestra de agua. Para el caso de agua 
para uso humano (no bebible), son las plantas potabilizadoras las encargadas de someter a procesos de tratamiento al agua que será posteriormente distribuida para consumo humano. En dichas instituciones, no sólo se realizan de 
forma continua los procesos de potabilización, sino que además tienen la posibilidad de contar con laboratorios para realizar análisis de Calidad del Agua, con el fin de validar los procesos que llevan a cabo, así como también
identificar problemas de calidad externos que se presenten en los cuerpos de agua ya disponibles para distribución y consumo humano.

El presente trabajo expone los resultados de un conjunto de muestreos y caracterizaciones de parámetros de Calidad del Agua en una de las plantas de potabilización del OOAPAS en la ciudad de Morelia. Se realizan análisis estadísticos 
de correlación para determinar los parámetros significativos para el proceso de potabilización, lo cual permite conocer las variables que definen los cambios generados en el agua al someterla al proceso de tratamiento. Una vez que se 
tienen identificados los parámetros de Calidad del Agua significativos, con el conjunto de datos de estas variables se entrenan algoritmos de clasificación basados en Redes Neuronales Artificiales, implementados en el lenguaje de 
programación Python, con el fin de generar un modelo basado en Inteligencia Artificial con la capacidad de generar estimaciones de Calidad del Agua en una muestra específica, mediante el uso de los valores dados para la muestra de 
los parámetros identificados como significativos.
