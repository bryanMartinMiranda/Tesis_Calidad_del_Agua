
\chapter*{Prólogo}
\addcontentsline{toc}{chapter}{Prólogo}

Los documentos de tesis son un legado escrito que prevalece por los años. Este documento, dependiendo de la institución que lo emite, puede variar en la forma que se presenta. En ese mismo sentido, la estructura ha sido definida en los lineamientos emitidos por la Dirección General de Educación Superior Tecnológica (DGEST). Sin embargo, la falta de actualización, la complejidad de manejar documentos de gran capacidad por procesadores como Microsoft Word y una mayor uniformidad de tesis en la División de Estudios de Posgrado e Investigación (DEPI). Han propiciado que el Comité Institucional de Posgrado e Investigación emitan esta primera versión de un manual/formato para la estructura de documentos como tesis, tesinas o disertación.\par 

En esta guía/formato se presentan e identifican los elementos y tipografía basados en el ISO-7144 y la DGEST, así como su implementación en lenguaje \LaTeX, éste a su vez funciona como una guía para que el autor pueda redactar y estructurar adecuadamente las partes del documento.\par

La plantilla está configurada para ejecutarse con cualquiera de las distribuciones libres de \LaTeX{} como MiK\TeX{} o \TeX Live, además ha sido probada en \textbf{Overleaf}, \textbf{\TeX Pad} y muestra absoluta compatibilidad con \textbf{Mendeley} y \textbf{Plot.ly}; el código está ampliamente basado en la plantilla \textbf{``Classic Thesis Template''} del autor \textbf{Andre Miede}.\par 

\bigskip
\begin{flushright}
-- \myName --	
\end{flushright}