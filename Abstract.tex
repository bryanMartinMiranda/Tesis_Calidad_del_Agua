%!TeX root=../thesisStructure.tex
\chapter*{Abstract} % Abstract for master degree thesis
\addcontentsline{toc}{chapter}{Abstract}

Worldwide, there is a permanent health problem related to the lack of quality water for human consumption. This problem is directly related to deficiencies in water treatment processes, in cases where the corresponding health 
regulations are not met.

Water Quality refers to the ranges of values required for the physicochemical and bacteriological parameters that influence the contamination levels of a water sample.
In the case of water for human use (non-drinking), water treatment plants are responsible for treating the water that will later be distributed for human consumption.
These institutions not only carry out continuous purification processes, but also have laboratories for water quality analysis, validating the processes they carry out and identifying external quality problems.

This paper presents the results of a series of sampling and characterizations of water quality parameters at one of the OOAPAS water treatment plants in the city of Morelia.
Statistical correlation analyses are performed to determine the significant parameters for the purification process, which allows us to identify the variables that define the changes generated in the water when it undergoes the treatment process.
Once the significant water quality parameters have been identified, classification algorithms based on artificial neural networks, implemented in the Python programming language, are trained with the data set for these variables. 
This is done to generate an artificial intelligence-based model capable of generating water quality estimates for a specific sample, using the values given for the parameters identified as significant.
