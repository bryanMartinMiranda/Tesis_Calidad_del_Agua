\chapter{Programación del Equipo Wi-Fi Pool Kit de Atlas Scientific}

El código implementado para el funcionamiento del equipo Wi-Fi Pool Kit en los muestreos realizados es el siguiente.

\begin{lstlisting}[language=C++]
    #include <iot_cmd.h>
    #include <WiFi.h>   // Include WiFi library 
    #include "ThingSpeak.h" // Include thingspeak library
    #include <sequencer4.h>  
    #include <sequencer1.h>  
    #include <Ezo_i2c_util.h> // I2C Modules 
    #include <Ezo_i2c.h>    // I2C Modules
    #include <Wire.h>

    WiFiClient client; // This device connects to a Wi-Fi network

    // Wi-Fi Credentials
    const String ssid = "sedeam"; // Name of the Wi-Fi network
    const String pass = "Sede@m2025"; // WiFi password
    const long myChannelNumber = 2399780; // Thingspeak channel number
    const char * myWriteAPIKey = "4SYBSZ1SZKPDQ8RD";    // ThingSpeak API Key

    // Create objects for boards
    Ezo_board DO = Ezo_board(97, "DO"); // DO circuit object, address is 97 and name is "DO"
    Ezo_board ORP = Ezo_board(98, "ORP");   // ORP circuit object, address is 98 and name is "ORP"
    Ezo_board RTD = Ezo_board(102, "RTD");  // RTD circuit object, address is 102 and name is "RTD"
    Ezo_board PMPL = Ezo_board(109, "PMPL");    // PMPL circuit object, address is 109 and name is "PMPL"
    Ezo_board EC = Ezo_board(100, "EC");    // EC circuit object, address is 100 and name is "EC"
    
    // Array of boards used for sending commands to specific boards
    Ezo_board device_list[] = {DO,ORP,RTD,EC,PMPL};

    Ezo_board* default_board = &device_list[0];

    const uint8_t device_list_len = sizeof(device_list) / sizeof(device_list[0]);

    // Enable pins for each circuit
    const int EN_DO = 12;
    const int EN_ORP = 27;
    const int EN_RTD = 15;
    const int EN_AUX = 33;

    const unsigned long reading_delay = 1000;   // How long we wait to receive a response, in milliseconds
    const unsigned long thingspeak_delay = 15000;   // How long we wait to send values to thingspeak, in milliseconds

    unsigned int poll_delay = 2000 - reading_delay * 2 - 300;

    #define PUMP_BOARD        PMPL
    #define PUMP_DOSE         10
    #define EZO_BOARD         DO        
    #define IS_GREATER_THAN   true      
    #define COMPARISON_VALUE  10

    float k_val = 0;    // Holds the k value for determining what to print in the help menu

    bool polling  = true;   // Variable to determine whether or not were polling the circuits
    bool send_to_thingspeak = true; // Variable to determine whether or not were sending data to thingspeak

    // Function to check if wifi is connected
    bool wifi_isconnected() 
    {                           
        return (WiFi.status() == WL_CONNECTED);
    }

    // Function to reconnect wifi if its not connected
    void reconnect_wifi() 
    {                                   
        if (!wifi_isconnected()) 
        {
            WiFi.begin(ssid.c_str(), pass.c_str());
            Serial.println("connecting to wifi");
        }
    }

    void thingspeak_send() 
    {
        if (send_to_thingspeak == true) 
        {                                                    
            if (wifi_isconnected()) 
            {
                int return_code = ThingSpeak.writeFields(myChannelNumber, myWriteAPIKey);
                if (return_code == 200) 
                {
                    Serial.println("sent to thingspeak");
                } 
                else 
                {
                    Serial.println("couldnt send to thingspeak");
                }
            }
        }
    }

    void step1();
    void step2();
    void step3();
    void step4();
    Sequencer4 Seq(&step1,reading_delay,&step2, 300,&step3, reading_delay,&step4, poll_delay);

    Sequencer1 Wifi_Seq(&reconnect_wifi, 10000);

    Sequencer1 Thingspeak_seq(&thingspeak_send, thingspeak_delay);  // Sends data to thingspeak with the time determined by thingspeak delay

    void setup() 
    {
        // Set enable pins as outputs
        pinMode(EN_DO, OUTPUT);                                                         
        pinMode(EN_ORP, OUTPUT);
        pinMode(EN_RTD, OUTPUT);
        pinMode(EN_AUX, OUTPUT);
        // Set enable pins to enable the circuits
        digitalWrite(EN_DO, LOW);                
        digitalWrite(EN_ORP, LOW);
        digitalWrite(EN_RTD, HIGH);
        digitalWrite(EN_AUX, LOW);

        Wire.begin();       
        Serial.begin(9600); // Serial communication to the computer

        WiFi.mode(WIFI_STA);         
        ThingSpeak.begin(client); 
        Wifi_Seq.reset();         
        Seq.reset();
        Thingspeak_seq.reset();
    }

    void loop() 
    {
        String cmd;
        Wifi_Seq.run();
        if (receive_command(cmd)) 
        {
            polling = false;
            send_to_thingspeak = false;
            if (!process_coms(cmd)) 
            {
                process_command(cmd, device_list, device_list_len, default_board);
            }
        }

        if (polling == true) 
        {
            Seq.run();
            Thingspeak_seq.run();
        }
    }

    // Function that controls the pumps activation and output
    void pump_function(Ezo_board &pump, Ezo_board &sensor, float value, float dose, bool greater_than) 
    {
        if (sensor.get_error() == Ezo_board::SUCCESS) 
        {
            bool comparison = false;
            if (greater_than) 
            {
                comparison = (sensor.get_last_received_reading() >= value);
            } 
            else 
            {
                comparison = (sensor.get_last_received_reading() <= value);
            }

            if (comparison) 
            {
                pump.send_cmd_with_num("d,", dose);
                delay(100);
                Serial.print(pump.get_name());
                Serial.print(" ");
                char response[20];
                if (pump.receive_cmd(response, 20) == Ezo_board::SUCCESS) 
                {
                    Serial.print("pump dispensed ");
                } 
                else 
                {
                    Serial.print("pump error ");
                }

                Serial.println(response);
            } 
            else 
            {
                pump.send_cmd("x");
            }
        }
    }

    void step1() 
    {
        // Send a read command
        RTD.send_read_cmd();
    }

    void step2() 
    {
        RTD.receive_read_cmd();
        print_success_or_error(RTD, String(RTD.get_last_received_reading(), 2).c_str());
        
        // If the temperature reading has been received and it is valid
        if ((RTD.get_error() == Ezo_board::SUCCESS) && (RTD.get_last_received_reading() > -1000.0)) 
        
        {
            DO.send_cmd_with_num("T,", RTD.get_last_received_reading());
            EC.send_cmd_with_num("T,", RTD.get_last_received_reading());
            ThingSpeak.setField(1, String(RTD.get_last_received_reading(), 2)); // Assign temperature readings to the first column of thingspeak channel
        } 
        else 
        {
            // If the temperature reading is invalid
            DO.send_cmd_with_num("T,", 20.0);
            EC.send_cmd_with_num("T,", 25.0);   // Send default temp = 25 deg C to PH sensor
            ThingSpeak.setField(1, String(25.0, 2));    // Assign temperature readings to the first column of thingspeak channel
        }

        Serial.print(",");
    }

    void step3() 
    {
        // Send a read command.
        DO.send_read_cmd();
        ORP.send_read_cmd();
        EC.send_read_cmd();
    }

    void step4() 
    {
        DO.receive_read_cmd();
        print_success_or_error(DO, String(DO.get_last_received_reading(), 2).c_str());
        
        if (DO.get_error() == Ezo_board::SUCCESS) 
        {
            ThingSpeak.setField(2, String(DO.get_last_received_reading(), 2));  // Assign PH readings to the second column of thingspeak channel
        }

        Serial.print(",");
        
        ORP.receive_read_cmd();
        print_success_or_error(ORP, String(ORP.get_last_received_reading(), 2).c_str());

        // If the ORP reading was successful
        if (ORP.get_error() == Ezo_board::SUCCESS) 
        {
            ThingSpeak.setField(3, String(ORP.get_last_received_reading(), 0)); // Assign ORP readings to the third column of thingspeak channel
        }

        Serial.print(",");

        EC.receive_read_cmd();
        print_success_or_error(EC, String(EC.get_last_received_reading(), 2).c_str());
        
        if (EC.get_error() == Ezo_board::SUCCESS)
        {
            ThingSpeak.setField(4, String(EC.get_last_received_reading(), 0));
        }

        Serial.println();
        pump_function(PUMP_BOARD, EZO_BOARD, COMPARISON_VALUE, PUMP_DOSE, IS_GREATER_THAN);
    }

    void start_datalogging() 
    {
        polling = true;
        send_to_thingspeak = true;
        Thingspeak_seq.reset();
    }

    bool process_coms(const String &string_buffer) 
    {
        if (string_buffer == "HELP") 
        {
            print_help();
            return true;
        }
        else if (string_buffer.startsWith("DATALOG")) 
        {
            start_datalogging();
            return true;
        }
        else if (string_buffer.startsWith("POLL")) 
        {
            polling = true;
            Seq.reset();
            int16_t index = string_buffer.indexOf(',');
            
            if (index != -1) 
            {
                float new_delay = string_buffer.substring(index + 1).toFloat();
                float mintime = reading_delay * 2 + 300;
                
                if (new_delay >= (mintime / 1000.0)) 
                {
                    Seq.set_step4_time((new_delay * 1000.0) - mintime);
                } 
                else 
                {
                    Serial.println("delay too short");
                }

            }

            return true;

        }

        return false;
    }

    void print_help() 
    {
        Serial.println(F("Atlas Scientific I2C pool kit"));
        Serial.println(F("Commands:"));
        Serial.println(F("datalog      Takes readings of all sensors every 15 sec send to thingspeak "));
        Serial.println(F("             Entering any commands stops datalog mode."));
        Serial.println(F("poll         Takes readings continuously of all sensors"));
        Serial.println(F("                                                                           "));
        Serial.println(F("ph:cal,mid,7     calibrate to pH 7                                         "));
        Serial.println(F("ph:cal,low,4     calibrate to pH 4                                         "));
        Serial.println(F("ph:cal,high,10   calibrate to pH 10                                        "));
        Serial.println(F("ph:cal,clear     clear calibration                                         "));
        Serial.println(F("                                                                           "));
        Serial.println(F("orp:cal,225          calibrate orp probe to 225mV                          "));
        Serial.println(F("orp:cal,clear        clear calibration                                     "));
        Serial.println(F("                                                                           "));
        Serial.println(F("rtd:cal,t            calibrate the temp probe to any temp value            "));
        Serial.println(F("                     t= the temperature you have chosen                    "));
        Serial.println(F("rtd:cal,clear        clear calibration                                     "));
    }
    
\end{lstlisting}

Las primeras 8 líneas del código son la importación de los archivos de encabezado que vienen incluidos en la descarga de las librerías del sitio web de Atlas Scientific, ya que la mayor parte de las funciones llamadas a 
lo largo del código son desarrolladas por el fabricante en lenguaje C++.

Al ser un equipo con la capacidad de conexión a internet y comunicación con la plataforma en línea ThingSpeak de almacenamiento de datos recolectados a nivel remoto, en las líneas de código 12 a 16 se especifican tanto las 
contraseñas y nombre de la red de internet usadas en el laboratorio, así como los identificadores de usuario creados en dicha plataforma. Inicialmente en los primeros muestreos sí se hacía uso de esta herramienta ThingSpeak 
para almacenar la información recolectada, sin embargo, posteriormente se hicieron modificaciones al código para prescindir de esta plataforma y poder recolectar una mayor cantidad de datos debido a que los tiempos de retardo 
para poder enviar los datos a la plataforma hacía que fuera pequeña la cantidad de datos guardados. Por lo tanto, se generaron cambios en el código para poder imprimir los datos muestrados en la terminal del IDE Arduino, lo 
cual permitió poder recolectar mayor cantidad de valores dado que ya no se tenían los tiempo de espera necesarios de comunicación entre el equipo Pool Kit y la plataforma ThingSpeak.

Esta implementación está desarrollada bajo la metodología de Programación Orientada a Objetos; en las líneas de código 18 a 23 se inicializan objetos de tipo \textit{Ezo Board}, los cuales permiten crear un tipo de identificador 
para cada una de los circuitos de acondicionamiento de señal correspondiente a 1 de los parámetros muestreados; estos objetos reciben como parámetros de inicio el nombre que corresponde a la variable específica, además del 
valor numérico que corresponde a la dirección usada en la comunicación I2C entre los circuitos y microcontrolador, para que el microcontrolador pueda diferenciar entre los diferentes circuitos.

En las líneas 32 a 36 del código se declaran los valores numéricos de los pines del microcontrolador que estarán conectados con los circuitos de instrumentación de señal; en este caso, el circuito de oxígeno disuelto se 
conecta con el pin 12, el circuito de ORP con el pin 27, el circuito de temperatura va al pin número 15, y finalmente el circuito auxiliar restante se conecta al pin 33 del microcontrolador. Posteriormente, en las líneas 
38 y 39 de código se declaran los valores de los tiempos de espera usados para la comunicación entre la plataforma ThingSpeak y el equipo Pool Kit.

En la línea 55 de código se declara la función \textit{WifiIsConnected} que continuamente se encarga de verificar que haya conexión de internet estable en el equipo cuando se están enviando datos a la plataforma ThingSpeak.
Posteriormente, en la línea 61 a 68 de código la función \textit{reconnect wifi} se encarga de reestablecer la conexión a internet en caso de que previamente se haya detectado que se ha perdido. En las líneas 70 a 87 del 
código se declara la función \textit{thingspeak send} que se encarga de mostrar mensajes en terminal sobre el estado de la comunicación entre el equipo Pool Kit y la plataforma, específicamente informar al usuario si ha 
existido algún registro de datos muestreado que no ha podido ser enviado a la plataforma y por lo tanto se ha perdido.

En las líneas 89 a 93 de código se declaran los encabezados de las funciones \textit{step}, las cuales en su estructura, que se define en líneas posteriores del código, presentan los procedimientos generados para la lectura 
de las señales obtenidas por los sensores, así como su procesamiento, para finalmente poder obtener un valor numérico que represente la magnitud medida para los parámetros muestreados, ya que los sensores únicamente generan
pequeños voltajes eléctricos que deben ser convertidos a su equivalente magnitud para las variables muestreadas.

En las líneas de código 100 a 120 se escribe la función \textit{setup} que en los entornos Arduino sirve para realizar la configuración inicial del microcontrolador, la cual se ejecuta sólo 1 vez. En esta función se habilitan 
los pines de la tarjeta que tienen conexión con los circuitos de instrumentación para poder generar la comunicación necesaria para obtener las señales que representan los parámetros muestreados. Igualmente, en esta función 
\textit{setup} también se hacen llamadas a las funciones encargadas de iniciar la conexión a internet y la sincronización con la plataforma ThingSpeak cuando se desea hacer almacenamiento en línea de las información recolectada.

En las líneas de código 122 a 141 se escribe la función \textit{loop} que en los entornos Arduino es la que se ejecuta repetidamente logrando así el funcionamiento continuo de la tarjeta de microcontrolador. En este caso, 
internamente la función \textit{loop} contiene las llamadas a las funciones \textit{step} que se encargan de extraer las señales eléctricas de las sondas para poder hacer los cálculos necesarios y obtener las magnitudes 
numéricas de los parámetros estudiados. Al tener las funciones \textit{step} dentro de la función \textit{loop} nos aseguramos de que la recolección, almacenamiento e impresión de los datos en terminal se realicen de forma 
ininterrumpida.

En la línea 183 de código se presenta la estructura de la función \textit{step1}, la cual es 1 de las 4 funciones base para el funcionamiento secuencial del equipo Pool Kit. En esta caso, \textit{step1} se encarga de ejecutar 
un llamado a otra función que pide que se inicie la lectura de un dato de la variable de temperatura. Esto es indispensable que sea el primer paso llevado a cabo para un muestreo individual de los 4 parámetros disponibles 
en el equipo, ya que la temperatura es factor que influye en las mediciones hechas para los otros sensores, por lo cual es importante que el primer dato recolectado sea la temperatura. De esta forma, al tener el valor de la 
temperatura actual, al calcular los valores numéricos para las señales de los otros sensores es posible usar el valor de temperatura para realizar la compensación necesaria en los cálculos de las otras variables.

En la línea 189 de código se inicia la estructura de la función \textit{step2}, en la cual se recibe el dato numérico del valor de temperatura actual y se imprime en la terminal del IDE Arduino. Además de lo anterior, dicho 
valor actual de temperatura es almacenado temporalmente para poder realizar compensación para los sensores de oxígeno disuelto y conductividad eléctrica, en caso de que dichos sensores estén conectados al equipo para medición 
de dichos parámetros. Por último, en esta función también se hace el envío a la plataforma ThingSpeak del primer valor correspondiente a un registro de datos, que en este caso viene siendo la temperatura y faltaría el envío 
de otros 3 datos numéricos adicionales para completar 1 registro que correspondería a 1 muestreo individual.

En la línea 213 de código, se presenta la estructura de la función \textit{step3}, en la cual se hacen llamados a funciones externas para poder iniciar la lectura de los otros 3 sensores disponibles, que para este caso serían 
oxígeno disuelto, ORP y conductividad eléctrica. Posteriormente en la línea 221 del código, se inicia la estructura de la función \textit{step4}, la cual es la última en el funcionamiento secuencial ininterrupmido del equipo 
Pool Kit. En esta última función, lo que se realiza es el envío a la plataforma ThingSpeak de los datos de los 3 parámetros restantes, además de generar su impresión en la terminal del Arduino IDE.

Finalmente, en la última parte del código a partir de la línea 304 se desarrolla la función \textit{print help} la cual permite imprimir en la terminal del Arduino IDE los comandos de ayuda para el usuario acerca del manejo 
del equipo Pool Kit durante su funcionamiento. Al ser invocada esta función por el usuario desde la terminal, se imprime información sobre acciones específicas, tales como iniciar el muestreo de manera continua, detener 
el envío de datos a la plataforma ThingSpeak, comandos de calibración para el sensor de potencial de Hidrógeno, comando de calibración para el sensor de ORP y comandos de calibración para el sensor de temperatura.
